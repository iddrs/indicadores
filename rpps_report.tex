\documentclass[12pt, a4paper]{article}
\usepackage[T1]{fontenc}
\usepackage[utf8]{inputenc}
\usepackage[brazil]{babel}
\usepackage{graphicx}
\usepackage{float}
\usepackage[tmargin=2cm, bmargin=1.5cm, lmargin=2cm, rmargin=1.5cm]{geometry}
\usepackage{parskip}
\setlength{\parindent}{0pt}
\setlength{\parskip}{1ex plus 0.5ex minus 0.2ex}

\title{Indicadores Fiscais e Gerenciais do FPSM}
\author{Everton da Rosa - Contador - CRC RS-076595/o-3}
\input{"cache/data.tex"}

\begin{document}

\maketitle

\begin{abstract}
Este relatório tem o objetivo de apresentar os principais indicadores legais e gerenciais relativos ao \textbf{Fundo de Previdência dos Servidores Municipais (RPPS)} do Município de \textbf{Independência/RS}.
\end{abstract}

\tableofcontents

\section{Indicadores Gerenciais}

Nesta seção estão apresentados alguns indicadores de cunho gerencial com a finalidade de auxiliar a gestão do RPPS nas decisões que envolvam aspectos contábeis, orçamentários e financeiros.

\subsection{Receita Orçamentária}

A receita engloba a receita orçamentária e intra-orçamentária arrecadada pelo RPPS, em termos líquidos de suas deduções.

Os valores são apresentados em termos mensais e acumulados até o mês de referência, comparados aos valores efetivamente realizados no ano anterior e com os valores previstos na lei orçamentária anual.


\subsection{Receita Total}

\input{"cache/pm_receita_total_1.tex"}
\input{"cache/pm_receita_total_2.tex"}
\input{"cache/pm_receita_total_3.tex"}

\begin{figure}[H]
\center
\includegraphics[keepaspectratio,width=\textwidth]{cache/pm_receita_total_1}
\end{figure}

\begin{figure}[H]
\center
\includegraphics[keepaspectratio,width=\textwidth]{cache/pm_receita_total_2}
\end{figure}


\subsubsection[Contribuição dos Servidores]{Contribuição dos Servidores Ativos, Inativos e Pensionistas}

A contribuição dos servidores ativos, inativos e pensionistas é prevista em lei à alíquota de 14\% sobre o salário contribuição.Engloba toda a receita do FPSM, inclusive a intra-orçamentária, já líquida das deduções.

\input{"cache/rpps_receita_contrib_serv_1.tex"}

\input{"cache/rpps_receita_contrib_serv_2.tex"}

\begin{figure}[H]
\center
\includegraphics[keepaspectratio,width=\textwidth]{cache/rpps_receita_contrib_serv_1}
\end{figure}

\begin{figure}[H]
\center
\includegraphics[keepaspectratio,width=\textwidth]{cache/rpps_receita_contrib_serv_2}
\end{figure}

\begin{figure}[H]
\center
\includegraphics[keepaspectratio,width=\textwidth]{cache/rpps_receita_contrib_serv_3}
\end{figure}

\begin{figure}[H]
\center
\includegraphics[keepaspectratio,width=\textwidth]{cache/rpps_receita_contrib_serv_4}
\end{figure}

\subsubsection[Contribuição Patronal Normal]{Contribuição Patronal – alíquota normal}

A contribuição patronal normal está prevista em lei no percentual de 14\% sobre a totalidade do salário contribuição e corresponde à contrapartida do Município à contribuição dos servidores para o FPSM.

\input{"cache/rpps_receita_contrib_patronal_normal_1.tex"}

\input{"cache/rpps_receita_contrib_patronal_normal_2.tex"}

\begin{figure}[H]
\center
\includegraphics[keepaspectratio,width=\textwidth]{cache/rpps_receita_contrib_patronal_normal_1}
\end{figure}

\begin{figure}[H]
\center
\includegraphics[keepaspectratio,width=\textwidth]{cache/rpps_receita_contrib_patronal_normal_2}
\end{figure}

\begin{figure}[H]
\center
\includegraphics[keepaspectratio,width=\textwidth]{cache/rpps_receita_contrib_patronal_normal_3}
\end{figure}

\begin{figure}[H]
\center
\includegraphics[keepaspectratio,width=\textwidth]{cache/rpps_receita_contrib_patronal_normal_4}
\end{figure}

\subsubsection[Contribuição Patronal Suplementar]{Contribuição Patronal – alíquota suplementar}

A contribuição patronal suplementar é a destinada à amortização do passivo atuarial identificado em avaliação atuarial periódica e tem alíquota definida em lei de forma escalonada em cada exercício financeiro.

\input{"cache/rpps_receita_contrib_patronal_suplem_1.tex"}

\input{"cache/rpps_receita_contrib_patronal_suplem_2.tex"}

\begin{figure}[H]
\center
\includegraphics[keepaspectratio,width=\textwidth]{cache/rpps_receita_contrib_patronal_suplem_1}
\end{figure}

\begin{figure}[H]
\center
\includegraphics[keepaspectratio,width=\textwidth]{cache/rpps_receita_contrib_patronal_suplem_2}
\end{figure}

\begin{figure}[H]
\center
\includegraphics[keepaspectratio,width=\textwidth]{cache/rpps_receita_contrib_patronal_suplem_3}
\end{figure}

\begin{figure}[H]
\center
\includegraphics[keepaspectratio,width=\textwidth]{cache/rpps_receita_contrib_patronal_suplem_4}
\end{figure}

\subsection{Disponibilidades do RPPS}

Esta seção apresenta dados sobre as disponibilidades e investimentos do RPPS.


\subsubsection{Disponibilidades de Caixa do RPPS}

As disponibilidades de caixa e equivalentes são apresentadas englobando os recursos vinculados à previdência e os da taxa administrativa.

O saldo projetado apresentado considera o saldo inicial do período, somando-se a previsão de receitas e subtraindo-se a previsão de despesas conforme a programação financeira estabelecida.

\input{"cache/rpps_disponibilidades_1.tex"}

\begin{figure}[H]
\center
\includegraphics[keepaspectratio,width=\textwidth]{cache/rpps_disponibilidades_1}
\end{figure}

O que se pretende com essa demonstração é dar uma ideia do desempenho da gestão do RPPS em termos de geração de caixa, visto que uma de suas finalidades é justamente a geração de caixa para fazer frente aos benefícios futuros dos segurados.


\subsection{Investimentos}

A gestão dos investimentos é feita por comitê especialmente constituído para esta finalidade, bem como por servidor imbuído da função de gestor de investimentos, cabendo ao serviço de contabilidade o registro dos fatos que impactam esses ativos.

Com isso em mente, este relatório se abstém de formar opinião sobre esse tema, visto que a competência para tal é dos citados comitê e gestor de investimentos.


\subsubsection[Resultado dos Investimentos]{Valorização/Desvalorização dos Investimentos do RPPS}

Nesta subseção estão refletidas as variações de valor dos investimentos do RPPS.

A metodologia de registro é a seguinte:

No caso de valorização, ocorre o registro de receita orçamentária;

Havendo desvalorização, ocorre o registro orçamentário de dedução de receita até o montante em que o saldo acumulado da dedução de receita é igual ao saldo acumulado da receita orçamentária principal. A desvalorização que supere esse valor é registrada sem execução orçamentária.

Embora essa metodologia simplifique o controle das variações de valor dos investimentos, ela traz como consequência a eventual divergência entre a desvalorização e o registro de dedução de receita orçamentária.

\input{"cache/rpps_disponibilidades_resultado_1.tex"}

\input{"cache/rpps_disponibilidades_resultado_2.tex"}

\begin{figure}[H]
\center
\includegraphics[keepaspectratio,width=\textwidth]{cache/rpps_disponibilidades_resultado_1}
\end{figure}

\begin{figure}[H]
\center
\includegraphics[keepaspectratio,width=\textwidth]{cache/rpps_disponibilidades_resultado_2}
\end{figure}

\begin{figure}[H]
\center
\includegraphics[keepaspectratio,width=\textwidth]{cache/rpps_disponibilidades_resultado_3}
\end{figure}

\begin{figure}[H]
\center
\includegraphics[keepaspectratio,width=\textwidth]{cache/rpps_disponibilidades_resultado_4}
\end{figure}

\subsection{Despesas Orçamentárias}

As despesas orçamentárias do RPPS são apresentadas nesta seção com base nos valores empenhados mensal e acumulado, comparados aos valores correspondentes do ano anterior.


\subsubsection{Despesa Total}

Engloba a totalidade das despesas empenhadas pelo FPSM, inclusive as intra-orçamentárias.

\input{"cache/rpps_despesa_total_1.tex"}

\input{"cache/rpps_despesa_total_2.tex"}

\begin{figure}[H]
\center
\includegraphics[keepaspectratio,width=\textwidth]{cache/rpps_despesa_total_1}
\end{figure}

\begin{figure}[H]
\center
\includegraphics[keepaspectratio,width=\textwidth]{cache/rpps_despesa_total_1}
\end{figure}



\subsubsection{Despesa com Inativos e Pensionistas do RPPS}

Os valores ora apresentados dizem respeito às despesas com inativos e pensionistas que passaram a integrar o RPPS após a sua criação, onerando o fundo em capitalização (plano previdenciário).

\input{"cache/rpps_despesa_inativos_rpps_1.tex"}

\input{"cache/rpps_despesa_inativos_rpps_2.tex"}

\begin{figure}[H]
\center
\includegraphics[keepaspectratio,width=\textwidth]{cache/rpps_despesa_inativos_rpps_1}
\end{figure}

\begin{figure}[H]
\center
\includegraphics[keepaspectratio,width=\textwidth]{cache/rpps_despesa_inativos_rpps_2}
\end{figure}



\subsubsection{Despesa com Inativos e Pensionistas do Tesouro}

Com a criação do atual fundo de previdência, os inativos e pensionistas já existentes foram incorporados ao RPPS cabendo o seu custeio, a partir daí, ao RPPS. Não houve previsão de repasse ou compensação financeira por parte do Executivo da parcela de despesas desses inativos e pensionistas\footnote{Inteligência do art. 70 da Lei Municipal nº 1.701/2005.}.

Desta forma, a segregação orçamentária desse público se dá apenas para fins gerenciais e de transparência.

\input{"cache/rpps_despesa_inativos_tesouro_1.tex"}

\input{"cache/rpps_despesa_inativos_tesouro_2.tex"}

\begin{figure}[H]
\center
\includegraphics[keepaspectratio,width=\textwidth]{cache/rpps_despesa_inativos_tesouro_1}
\end{figure}

\begin{figure}[H]
\center
\includegraphics[keepaspectratio,width=\textwidth]{cache/rpps_despesa_inativos_tesouro_2}
\end{figure}



\subsubsection{Despesas Administrativas do RPPS}

São as despesas destinadas à manutenção do FPSM, inclusive sua estrutura de gestão e serviços de apoio administrativo.

\input{"cache/rpps_despesa_adm_1.tex"}

\input{"cache/rpps_despesa_adm_2.tex"}

\begin{figure}[H]
\center
\includegraphics[keepaspectratio,width=\textwidth]{cache/rpps_despesa_adm_1}
\end{figure}

\begin{figure}[H]
\center
\includegraphics[keepaspectratio,width=\textwidth]{cache/rpps_despesa_adm_2}
\end{figure}



\subsubsection{Despesas com Aposentadorias}

Engloba todas as aposentadorias pagas pelo FPSM.

\input{"cache/rpps_despesa_aposent_1.tex"}

\input{"cache/rpps_despesa_aposent_2.tex"}

\begin{figure}[H]
\center
\includegraphics[keepaspectratio,width=\textwidth]{cache/rpps_despesa_aposent_1}
\end{figure}

\begin{figure}[H]
\center
\includegraphics[keepaspectratio,width=\textwidth]{cache/rpps_despesa_aposent_2}
\end{figure}



\subsubsection{Despesas com Pensões}

Considera a totalidade das pensões a dependentes de segurados falecidos.

\input{"cache/rpps_despesa_pensoes_1.tex"}

\input{"cache/rpps_despesa_pensoes_2.tex"}

\begin{figure}[H]
\center
\includegraphics[keepaspectratio,width=\textwidth]{cache/rpps_despesa_pensoes_1}
\end{figure}

\begin{figure}[H]
\center
\includegraphics[keepaspectratio,width=\textwidth]{cache/rpps_despesa_pensoes_2}
\end{figure}



\subsection{Dívida do Executivo em Regime de Parcelamento}

O Poder Executivo mantém um parcelamento de contribuições e outros valores não repassados e de empréstimos feitos junto ao RPPS. Esse parcelamento foi autorizado em 360 parcelas sendo que para o ano de 2022 foram pagas as parcelas 203 a 214.

Como forma de acompanhamento dos valores recebidos e do saldo devedor, foi elaborada projeção atualizada pela estimativa do IPCA englobando todo o período do parcelamento.

Também são apresentados os valores efetivamente realizados.


\input{"cache/rpps_parcelamento_1.tex"}

\begin{figure}[H]
\center
\includegraphics[keepaspectratio,width=\textwidth]{cache/rpps_parcelamento_1}
\end{figure}

\begin{figure}[H]
\center
\includegraphics[keepaspectratio,width=\textwidth]{cache/rpps_parcelamento_2}
\end{figure}



\subsection{Compensação Previdenciária}

A compensação previdenciária entre o RPPS e o RGPS e outros RPPS tem a finalidade de transferir os recursos de contribuições dos segurados e patronal para os RPPS nos quais o segurado efetivamente se inativou ou gerou pensionistas.

Assim, demonstra-se a seguir o balanço entre os recursos recebidos de outros RPPS e do RGPS e aqueles transferidos a eles pelo FPSM.


\input{"cache/rpps_comp_prev_1.tex"}

\input{"cache/rpps_comp_prev_2.tex"}

\begin{figure}[H]
\center
\includegraphics[keepaspectratio,width=\textwidth]{cache/rpps_comp_prev_1}
\end{figure}

\begin{figure}[H]
\center
\includegraphics[keepaspectratio,width=\textwidth]{cache/rpps_comp_prev_2}
\end{figure}



\vspace{16pt}

\begin{center}
	Independência, RS, \today
\end{center}

\vspace{36pt}

\begin{center}
	EVERTON DA ROSA\\
	Contador\\
	CRC RS-076595/O-3
\end{center}
\end{document}
