
\subsection[Limite de Suplementação]{Limite de Suplementação Autorizado na Lei Orçamentária Anual}

Tendo por base o § 8º do art. 165 da Constituição Federal, a Lei Orçamentária Anual traz em seu artigo 7º a autorização para abertura de créditos suplementares no Poder Legislativo, \textit{limitado a 15\% da dotação inicial} desse Poder.

Por sua vez, o art. 8º da Lei Orçamentária Anual elenca situações as quais os créditos abertos com base na autorização do art. 7º da LOA não oneram esse limite.

Desta forma, a cada resolução de abertura de crédito amparada nessa autorização, é feito o acompanhamento dos valores onerados do limite do Legislativo.

\input{"cache/cm_suplem.tex"}

Cabe ressaltar que a partir do momento em que ocorrer a suplementação acima do limite permitido, isso representa o desatendimento a disposições legais e constitucionais, por isso é importantíssimo o acompanhamento constante desse indicador a fim de evitar eventual imputação de crime à Presidência da Câmara.

\begin{figure}[H]
\center
\includegraphics[keepaspectratio,width=\textwidth]{cache/cm_suplem}
\end{figure}

O \textit{índice estimado} o comprometimento com base na média mensal\footnote{Limite dividido por 12 meses multiplicado pelo número de meses decorridos.}.