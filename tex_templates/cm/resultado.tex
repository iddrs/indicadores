
\subsection[Resultado Orçamentário]{Resultado Orçamentário}

O Poder Legislativo não possui arrecadação própria, sendo suas despesas arcadas mediante recursos repassados pelo Poder Executivo na forma de duodécimos. Isso não significa que o Poder Legislativo possui algum tipo de relação de dependência para com o Executivo, vez que é uma imposição constitucionalmente estabelecida essa sistemática de financiamento.

\begin{quotation}
Art. 168. Os recursos correspondentes às dotações orçamentárias, compreendidos os créditos suplementares e especiais, destinados aos órgãos dos Poderes Legislativo e Judiciário, do Ministério Público e da Defensoria Pública, ser-lhes-ão entregues até o dia 20 de cada mês, em duodécimos, na forma da lei complementar a que se refere o art. 165, § 9º.
\end{quotation}

Nessa sistemática, o valor a ser repassado em cada exercício financeiro corresponde à dotação atualizada da Câmara de Vereadores, \textit{limitada a 7\% da Receita Efetivamente Realizada no Ano Anterior}\footnote{Vide indicador de Gasto Total}. Desta forma, e considerando a possibilidade de que as despesas efetivamente realizadas em cada ano sejam inferiores aos recursos efetivamente recebidos pelo Poder Legislativo, pode ocorrer uma situação de restarem ao final do ano recursos financeiros\footnote{Caso isso ocorra o § 2º do art. 168 da Constituição Federal determina a devolução ao Poder Executivo.}. 

Isto posto, o acompanhamento de eventual superávit estimado se constitui em importante indicador de gestão, sendo que esse resultado positivo entre duodécimo recebido e despesa executada, além de ser convertido ao final do exercício em recurso disponível para o Executivo, também pode ser utilizado durante o ano como fonte de recursos para abertura de créditos adicionais no Poder Executivo, mediante acordo entre os Poderes.

\input{"cache/cm_resultado.tex"}

Para este indicador, é apresentado a evolução mensal do ano corrente e do ano anterior para fins de comparação.

O cálculo dos valores estimados é feito tendo por base a despesa com folha de pagamento e auxílio-alimentação, diárias, passagens e indenizações de viagem e inscrições em curso\footnote{Média mensal empenhada extrapolada anualmente.},além de considerar as reduções orçamentárias que foram fonte de crédito adicional no Poder Executivo.
Para o mês de dezembro, em vez da estimativa, é utilizado o valor efetivamente apurado no ano.

\begin{figure}
\center
\includegraphics[keepaspectratio,height=1\linewidth]{cache/cm_resultado}
\end{figure}

O \textit{índice estimado} o comprometimento com base na média mensal\footnote{Limite dividido por 12 meses multiplicado pelo número de meses decorridos.}.