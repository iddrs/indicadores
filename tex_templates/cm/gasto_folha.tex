
\subsection[Gastos com Folha de Pagamentos]{Limite Constitucional de Gasto com Folha de Pagamentos}

A Constituição Federal, no § 1ºdo art. 29-A limita o gasto com folha de pagamento a \textit{70\% do limite de gasto total} do Poder Legislativo.

\begin{quotation}
Art. 29-A [...]
 
§ 1º A Câmara Municipal não gastará mais de setenta por cento de sua receita com folha de pagamento, incluído o gasto com o subsídio de seus Vereadores.
\end{quotation}

Segundo as disposições do TCE/RS, além do grupo de natureza de despesa de Pessoal e Encargos Sociais, computam nesse limite as despesas relativas a auxílio-alimentação (independentemente do nome dado ao auxílio) quando não empenhadas naquele grupo de natureza de despesa.

\input{"cache/cm_gasto_folha.tex"}

A verificação do atendimento ou não a esse limite constitucional é feito apenas no encerramento do exercício, o que equivale dizer que o acompanhamento mensal desse indicador tem caráter gerencial.

\begin{figure}[H]
\center
\includegraphics[keepaspectratio,width=\textwidth]{cache/cm_gasto_folha}
\end{figure}

