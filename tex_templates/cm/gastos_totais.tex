
\subsection[Gastos Totais]{Limite Constitucional de Gastos Totais}

O Poder Legislativo possui um limite de gasto total anual, este expresso no art. 29-A da Constituição Federal.

\begin{quotation}
Art.  29-A.  O total da despesa do Poder Legislativo Municipal, incluídos os subsídios dos Vereadores e excluídos os gastos com inativos, não poderá ultrapassar os seguintes percentuais, relativos ao somatório da receita tributária e das transferências previstas no § 5º do art. 153 e nos arts. 158 e 159, efetivamente realizado no exercício anterior:
\end{quotation}

No caso do Legislativo de Independência, o \textit{limite constitucional é de 7\%} da \textit{Receita Efetivamente Arrecadada no Exercício Anterior}.

\input{"cache/cm_gasto_total.tex"}

É importante salientar que o indicador de gastos totais é apurado apenas no encerramento do exercício para fins de cumprimento ou não do limite constitucional, sendo que a sua estimativa mensal tem finalidade gerencial.

\begin{figure}[H]
\center
\includegraphics[keepaspectratio,height=1\linewidth]{cache/cm_gasto_total}
\end{figure}

A estimativa de gastos totais do Legislativo é baseada na dotação atualizada da Câmara de Vereadores.
