
\subsubsection{Despesa com Inativos e Pensionistas do Tesouro}

Com a criação do atual fundo de previdência, os inativos e pensionistas já existentes foram incorporados ao RPPS cabendo o seu custeio, a partir daí, ao RPPS. Não houve previsão de repasse ou compensação financeira por parte do Executivo da parcela de despesas desses inativos e pensionistas\footnote{Inteligência do art. 70 da Lei Municipal nº 1.701/2005.}.

Desta forma, a segregação orçamentária desse público se dá apenas para fins gerenciais e de transparência.

\input{"cache/rpps_despesa_inativos_tesouro_1.tex"}

\input{"cache/rpps_despesa_inativos_tesouro_2.tex"}

\begin{figure}[H]
\center
\includegraphics[keepaspectratio,width=\textwidth]{cache/rpps_despesa_inativos_tesouro_1}
\end{figure}

\begin{figure}[H]
\center
\includegraphics[keepaspectratio,width=\textwidth]{cache/rpps_despesa_inativos_tesouro_2}
\end{figure}

