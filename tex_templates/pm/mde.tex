
\section[Aplicação de Recursos em MDE]{Aplicação de Recursos de Impostos e Transferências de Impostos em Manutenção e Desenvolvimento do Ensino}

A Constituição Federal estabeleceu que os Municípios devem aplicar um mínimo de 25\% da sua receita de impostos e transferências decorrentes de impostos na manutenção e desenvolvimento do ensino (MDE).

\begin{quotation}
Art. 212. A União aplicará, anualmente, nunca menos de dezoito, e os Estados, o Distrito Federal e os Municípios vinte e cinco por cento, no mínimo, da receita resultante de impostos, compreendida a proveniente de transferências, na manutenção e desenvolvimento do ensino.
 
Em decorrência desse mandamento constitucional, mensalmente é apurado o respectivo índice de gastos com fins de acompanhamento, sendo que é o índice ao final do exercício o que determina o adimplemento ou não da regra constitucional.
\end{quotation}

\input{"cache/pm_mde_1.tex"}

O índice é apurado de acordo com as disposições do TCE/RS.

\begin{figure}[H]
\center
\includegraphics[keepaspectratio,width=\textwidth]{cache/pm_mde_1}
\end{figure}

