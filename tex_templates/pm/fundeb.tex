
\section[Aplicação de Recursos do FUNDEB na Remuneração]{Aplicação dos Recursos Recebidos do FUNDEB na Remuneração dos Profissionais da Educação}

O FUNDEB constitui um importante meio de financiamento da educação pública brasileira e tem, entre outras, a função de promover a valorização dos profissionais da educação, em especial o Magistérios.
 
Nesse condão, a Constituição Federal determina que um mínimo de 70\% dos recursos recebidos do FUNDEB em cada ano seja aplicado na remuneração dos profissionais da educação.

\begin{quotation}
Art. 212-A. Os Estados, o Distrito Federal e os Municípios destinarão parte dos recursos a que se refere o caput do art. 212 desta Constituição à manutenção e ao desenvolvimento do ensino na educação básica e à remuneração condigna de seus profissionais, respeitadas as seguintes disposições: (Incluído pela Emenda Constitucional nº 108, de 2020)    Regulamento

[...]

XI - proporção não inferior a 70\% (setenta por cento) de cada fundo referido no inciso I do caput deste artigo, excluídos os recursos de que trata a alínea "c" do inciso V do caput deste artigo, será destinada ao pagamento dos profissionais da educação básica em efetivo exercício, observado, em relação aos recursos previstos na alínea "b" do inciso V do caput deste artigo, o percentual mínimo de 15\% (quinze por cento) para despesas de capital; (Incluído pela Emenda Constitucional nº 108, de 2020)
\end{quotation}

Em decorrência desse mandamento constitucional, mensalmente é apurado o respectivo índice com fins de acompanhamento, sendo que é o índice ao final do exercício o que determina o adimplemento ou não da regra constitucional.

\input{"cache/pm_fundeb_1.tex"}

O índice é apurado de acordo com as disposições do TCE/RS.

\begin{figure}[H]
\center
\includegraphics[keepaspectratio,width=\textwidth]{cache/pm_fundeb_1}
\end{figure}

