
\section[Aplicação de Recursos em ASPS]{Aplicação de Recursos de Impostos e Transferências de Impostos em Ações e Serviços Públicos de Saúde}

A Lei Complementar nº 141/2012 estabeleceu que os Municípios devem aplicar um mínimo de 15\% da sua receita de impostos e transferências decorrentes de impostos em ações e serviços públicos em saúde (ASPS).

\begin{quotation}
Art. 7º Os Municípios e o Distrito Federal aplicarão anualmente em ações e serviços públicos de saúde, no mínimo, 15\% (quinze por cento) da arrecadação dos impostos a que se refere o art. 156 e dos recursos de que tratam o art. 158 e a alínea “b” do inciso I do caput e o § 3º do art. 159, todos da Constituição Federal.
\end{quotation}

Em decorrência desse mandamento legal, mensalmente é apurado o respectivo índice de gastos com fins de acompanhamento, sendo que é o índice ao final do exercício o que determina o adimplemento ou não da regra constitucional.

\input{"cache/pm_asps_1.tex"}

O índice é apurado de acordo com as disposições do TCE/RS.

\begin{figure}[H]
\center
\includegraphics[keepaspectratio,width=\textwidth]{cache/pm_asps_1}
\end{figure}

