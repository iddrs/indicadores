
\section[Limite da Despesa Corrente vs Receita Corrente]{Limite da Despesa Corrente como Proporção da Receita Corrente}

A Constituição Federal impôs vedações aos entes que, num exercício financeiro, apresentarem uma relação entre despesa e receita correntes superior a 95%.
 
\begin{quotation}
Art. 167-A. Apurado que, no período de 12 (doze) meses, a relação entre despesas correntes e receitas correntes supera 95\% (noventa e cinco por cento), no âmbito dos Estados, do Distrito Federal e dos Municípios, é facultado aos Poderes Executivo, Legislativo e Judiciário, ao Ministério Público, ao Tribunal de Contas e à Defensoria Pública do ente, enquanto permanecer a situação, aplicar o mecanismo de ajuste fiscal de vedação da: (Incluído pela Emenda Constitucional nº 109, de 2021)

[...]

§ 1º Apurado que a despesa corrente supera 85\% (oitenta e cinco por cento) da receita corrente, sem exceder o percentual mencionado no caput deste artigo, as medidas nele indicadas podem ser, no todo ou em parte, implementadas por atos do Chefe do Poder Executivo com vigência imediata, facultado aos demais Poderes e órgãos autônomos implementá-las em seus respectivos âmbitos. (Incluído pela Emenda Constitucional nº 109, de 2021)
\end{quotation}

Desta forma, esta seção apresenta o cálculo considerando a administração consolidada (Executivo, RPPS e Legislativo), conforme a metodologia do TCE/RS.

\input{"cache/pm_receita_despesas_limite_1.tex"}


\begin{figure}[H]
\center
\includegraphics[keepaspectratio,width=\textwidth]{cache/pm_receita_despesas_limite_1}
\end{figure}
